% place all figures in a sub-folder called figures, located in the folder where
% the main latex file is
%

%
%Packages
%
\usepackage{enumitem}
\usepackage{xspace}
\usepackage{fancyhdr}
\usepackage{afterpage}
\usepackage[usenames,dvipsnames]{xcolor}
\usepackage{dcolumn}
\usepackage{comment}
\usepackage{listings}
\usepackage{makeidx}
\usepackage{longtable}
\usepackage{chemformula}
\usepackage[utf8]{inputenc}
\usepackage{comment}
\usepackage{subfiles}

%get the floats right
\usepackage[dvips]{epsfig}
\usepackage{epstopdf}
\usepackage{graphicx}% Include figure files

%math packages for more advanced math
\usepackage{mathtools}
\usepackage{amsmath}
\usepackage{amssymb}
\usepackage{bm}
\usepackage{amsfonts}
\usepackage{commath} %http://ctan.mackichan.com/macros/latex/contrib/commath/commath.pdf
%\usepackage[a4paper, total={6in, 9in}]{geometry}

% then when you include the graphics, do not put any path there
% e.g., \includegraphics[width=0.8\textwidth]{<name>}
% for a figure called <name>.pdf (or <name>.png, or <name>.jpg...latex will do
% its best to choose the extension from known image file types).
%
% as preference use pdf format files.  These are vector graphics and so produce
% higher quality images at all scales.
\graphicspath{ {./figures/} }

