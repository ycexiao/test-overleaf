%------------------------------------------------------------------------------
% Template file for the basic latex article class
\documentclass{article}

\usepackage[usenames,dvipsnames]{xcolor}
\usepackage{comment}
\usepackage[utf8]{inputenc}
\usepackage{subfiles}
\usepackage[dvips]{epsfig}
\usepackage{epstopdf}
\usepackage{graphicx}% Include figure files
\usepackage{mathtools}
\usepackage{amsmath}
\usepackage{amssymb}
\usepackage{bm}
\usepackage{amsfonts}
% place all figures in a sub-folder called figures, located in the folder where
% the main latex file is

%Packages
%

%get the floats right

%math packages for more advanced math


% then when you include the graphics, do not put any path there
% e.g., \includegraphics[width=0.8\textwidth]{<name>}
% for a figure called <name>.pdf (or <name>.png, or <name>.jpg...latex will do
% its best to choose the extension from known image file types).
%
% as preference use pdf format files.  These are vector graphics and so produce
% higher quality images at all scales.
\graphicspath{ {./figures/} }


%units
\newcommand{\iaa}{\AA\ensuremath{^{-1}}\xspace}
\newcommand{\iaasquared}{\AA\ensuremath{^{-2}}\xspace}
\newcommand{\aasquared}{\AA\ensuremath{^{2}}\xspace}

% comments
% Add your own here. Pick your own color.
\newcommand{\sjb}[1]{\textcolor{blue}{[sjb:#1]}}

% correctly formated latin
\newcommand{\etal}{\emph{et al.}\xspace}
\newcommand{\adhoc}{\emph{ad hoc}\xspace}
\newcommand{\insitu}{\emph{in situ}\xspace}
\newcommand{\operando}{\emph{operando}\xspace}


% Colored symbols used during editing.
% Also, define new colors here if you like
\definecolor{dgreen}{HTML}{008000}
\newcommand{\done}{\textcolor{dgreen}{{\large{$\checkmark$}}}}
\newcommand{\doing}{\textcolor{blue}{{\large{$\blacklozenge$}}}}
\newcommand{\checkme}{\textcolor{orange}{{\large{$\blacktriangleleft$}}}}
\newcommand{\fixme}{\textcolor{red}{{\large{$\blacktriangleright$}}}}

% some style preferences for code listings
\definecolor{codegreen}{rgb}{0,0.6,0}
\definecolor{codegray}{rgb}{0.5,0.5,0.5}
\definecolor{codepurple}{rgb}{0.58,0,0.82}
\definecolor{backcolour}{rgb}{0.95,0.95,0.92}
\lstdefinestyle{mystyle}{
    backgroundcolor=\color{backcolour},
    commentstyle=\color{codegreen},
    keywordstyle=\color{magenta},
    numberstyle=\tiny\color{codegray},
    stringstyle=\color{codepurple},
    basicstyle=\ttfamily\footnotesize,
    breakatwhitespace=false,
    breaklines=true,
    captionpos=b,
    keepspaces=true,
    numbers=left,
    numbersep=5pt,
    showspaces=false,
    showstringspaces=false,
    showtabs=false,
    tabsize=2
}
\lstset{style=mystyle}


\title{A Small \LaTeX{} Article Template\thanks{To your mother}}
\author{Your Name  \\
	Your Company / University  \\
	\and
	The Other Dude \\
	His Company / University \\
	}

\date{\today}
% Hint: \title{what ever}, \author{who care} and \date{when ever} could stand
% before or after the \begin{document} command
% BUT the \maketitle command MUST come AFTER the \begin{document} command!
\begin{document}

\maketitle


\begin{abstract}
Short introduction to subject of the paper \ldots
\end{abstract}

\section{Introduction}
Make it possible for all to write documents with \LaTeX{}!
Thanks to https://www.namsu.de/Extra/klassen/latex-article-template.html
for the template

\paragraph{Outline}
First we start with a little example of the article class, which is an
important documentclass. But there would be other documentclasses like
book \ref{book}, report \ref{report} and letter \ref{letter} which are
described in Section \ref{documentclasses}. Finally, Section
\ref{conclusions} gives the conclusions.



\section{Documentclasses} \label{documentclasses}

\begin{itemize}
\item article
\item book
\item report
\item letter
\end{itemize}


\begin{enumerate}
\item article
\item book
\item report
\item letter
\end{enumerate}

\begin{description}
\item[article\label{article}]{Article is \ldots}
\item[book\label{book}]{The book class \ldots}
\item[report\label{report}]{Report gives you \ldots}
\item[letter\label{letter}]{If you want to write a letter.}
\end{description}


\section{Conclusions}\label{conclusions}
There is no longer \LaTeX{} example which was written by \cite{dirac1928}.

\bibliography{{{ cookiecutter.project_name }}.bib,user-default.bib}
\bibliographystyle{chicago}


\end{document}